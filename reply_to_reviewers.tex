\documentclass[11pt, a4paper]{article}

\usepackage[tmargin=20 mm,lmargin=25 mm,rmargin=25 mm,bmargin=25 mm]{geometry}
\usepackage[parfill]{parskip}
\usepackage{graphicx}
\usepackage{amsmath}
\usepackage{amsfonts}
\usepackage{amssymb}
\usepackage{latexsym}
\usepackage[hidelinks]{hyperref}
\usepackage{enumerate}
\usepackage{overpic}
\usepackage{mathrsfs}

%\usepackage{times}
\usepackage{xcolor} %colori
\newcommand{\corrB}[1]{{\color{blue} #1}}  %comandi per correzioni
\newcommand{\corrR}[1]{{\color{red} #1}}            %comandi per correzioni
\newcommand{\corrV}[1]{{\color{green} #1}}          %comandi per correzioni

% \newcommand{\comment}[1]{\textit{#1}}
% \newcommand{\reply}[1]{{\color{blue}{#1}}}

\newcommand{\todo}[1]{\textcolor{blue}{#1}}
\newcommand{\highlight}[1]{\textcolor{red}{#1}}
\newcommand{\comment}[1]{{%
\begin{sffamily}\leavevmode\color{black}#1
\end{sffamily}}}%


\def\baselinestretch{1.1}
\def\vv{\vspace{\baselineskip}}
\def\v{\vspace{2mm}}

\begin{document}

\raisebox{-\height}[0mm][0mm]{\includegraphics[scale=0.4]{ox_brand_cmyk_pos_rect}}

\hspace{105mm}%
\raisebox{-7.7mm}[0mm][0mm]{\begin{tabular}{@{}l@{}}
Department of Engineering Science\\
University of Oxford\\
Oxford, OX1 3PJ, UK\\
% \begin{tabular}[t]{@{}ll@{}}
% tel: & +44 1865 273000\rule{0pt}{12pt}\\ 
% e-mail: & \texttt{mark.cannon@eng.ox.ac.uk}
% \end{tabular}
{mark.cannon@eng.ox.ac.uk}
\\
\today\rule{0pt}{25pt}
\end{tabular}}

\vspace{30mm}

\begin{tabular}{@{}l@{}} 
Dr.~Kenneth Clarkson\\
Co-Editor-in-Chief\\
Discrete \& Computational Geometry
\end{tabular}

\vspace{15mm}

\noindent Dear Dr.\ Clarkson,

\textbf{Manuscript ID DCGE-D-17-00022, ``Explicit Operations on Parametrised Polyhedra''\\
by Rainer Manuel Schaich and Mark Cannon}
 
Please find attached a revised manuscript taking into account the reviewers' comments. Changes are in blue to facilitate identification.

In particular:

\begin{itemize}
\item[--] We provide better motivation for the idea of the parametric Pontryagin difference in the Introduction, and give clearer and more intuitive explanations for our results throughout the paper.
\item[--] We have corrected Definition 3 (previously called Definition 2) and re-written Lemma 1 (previously called Theorem 1) to address the reviewers' concerns about the conditions for parametric convexity, and we have corrected Theorem 3 (previously called Corollary 2) according to the suggestion of the second reviewer. We have likewise simplified and clarified the proof of Theorem 4 (previously called Corollary 3), as requested by the second reviewer.
\item[--] We have improved the clarity and precision of the text throughout the paper.
\end{itemize}

%replies to the editor's and reviewers' reports  
The attached replies to the reviewers give detailed point-by-point responses to the review comments.

Yours sincerely,

\includegraphics[scale=0.08]{MRCannon_black_signature.jpg}

Mark Cannon

\newpage


% \vv

% \centerline{\rule{0.9\textwidth}{0.25pt}}

\subsection*{Review 1}

\textbf{Comment 1.1:}
\comment{[Overview] This paper introduces parametric convexity as a property of set-valued maps and shows how it can be used for set-valued maps whose images are polyhedra.
I really like the concept of parametric convexity, this is something new under the sun. However, the writeup needs a substantial revision for this paper to become suitable for DCG. The paper is missing a story, and statements as well as proofs tend to be unclear. I also have doubts concerning the correctness of the main result, as it builds on a technical result (Theorem 1) that seems false, at least in the way it is presented now.
I therefore recommend to reject the paper, but ask the authors to re-submit, should they be able to fix the problems indicated below and tell a coherent story around it.
Here is the story that I would tell after reading the paper. Maybe the authors can draw some inspiration from it\dots}

\v
\textbf{Response 1.1:}
We would like to thank the reviewer for this overall positive assessment and for providing helpful and constructive suggestions for improving the paper. 

We have acted on the reviewer's advice to improve the motivation for the paper with a simple example illustrating the concept of the parametric Pontryagin difference. This is now included in the Introduction.  

We have clarified a number of statements throughout the paper, introducing Definition 1, correcting Definition 3 and re-writing Lemma 1 and Theorems 3 and 4. We have also simplified and clarified the proof of Theorem 4 by introducing two new lemmas (Lemma 2 and Lemma~3). 

\v
\textbf{Comment 1.2:}
\comment{
[Characterization of parametric convexity?] Theorem 1 (which should
actually be a lemma, as it is a technical tool) claims to give a characterization 
of parametric convexity in the case where $P$ is continuous and all ``slices'' $P(y)$ are convex\dots In fact, I didn’t understand the proof in the paper at all. Therefore, I also could not identify a version of Theorem 1 that is true. The issue is somewhat severe, since several corollaries and also the main result are drawn from this characterization.
}

\v
\textbf{Response 1.2:}
The revised the paper includes corrections to Definition 3 (previously called Definition 2) and Lemma 1 (previously called Theorem 1) that address the reviewer's concerns about the characterization of parametric convexity. In particular, the statement of Lemma 1 now includes the condition that $\mathcal W(p)$ must contain $0$, which prevents the particular problem pointed out by the reviewer for the case in which $\mathcal W(p)$ is a singleton for all $p$, and the proof of Lemma 1 has been entirely re-written.

\v
\textbf{Comment 1.3:}
\comment{
Generally, you need to be more precise. An example is Corollary 3. Apart from the fact that it is a bit unusual to call the central result ``Corollary 3'', it has the following issues: When you say that the number of vertices and rays is constant, this can mean several things. It can mean $O(1)$, or it can mean some fixed number (which is what you need). Also, you should say what the $b_{i,k}$ and $c_{i,k}$ are, currently, they are just ``free variables''. Also, if you say, ``within any partition'', you mean ``within any member of the partition''. When you say ``polytopic'', you mean ``polyhedric'', and whatever you say, it should be defined what you mean.}

\v
\textbf{Response 1.3:}
Corollary~3 is replaced with Theorem~4 in the revised paper. In this and the supporting Lemmas 2 and 3, we have been careful to state that the number of vertices is assumed to be equal to a given integer rather than simply constant. The revised paper explains the parameters that define the polyhedral maps in Theorem 3 and Theorem 4 (which is now stated in terms of the map given in equation (4.2)), and gives their dimensions. We have corrected the references to the partition of $Y$ and have replaced the term ``polytopic'' with ``polyhedral'' throughout the paper. 

\v
\textbf{Comment 1.4:}
\comment{
p5, l16: (3.2) does not require the graph to be non-convex. If the graph was a hyperplane, say, you could still have (3.2).}

\v
\textbf{Response 1.4:}
This has been corrected in the revised paper (last paragraph on p.5).

\v
\textbf{Comment 1.5:}
\comment{
p3, l33: when you say that you use the Perles conjecture, you should (a) mention that the conjecture has has been proved, and (b) cite Blind and Mani who first proved it. At least the version you need, there does not seem to be a standard meaning of the Perles conjecture.}

\v
\textbf{Response 1.5:}
Done (in the second paragraph on p.4 and the proof of Corollary 1 on p.11).

\v
\textbf{Comment 1.6:}
\comment{
p4, l47-49: this implication is unclear to me. If the Minkowski sum $\lambda \mathcal W (p_1) \oplus (1 - \lambda)\mathcal W(p_2)$ is linearly separable from the slice $\mathcal W(\lambda y_1 + (1 - \lambda)y_2)$, then I see no argument here. Even if this is not the case, one would have to argue more.}

\v
\textbf{Response 1.5:}
We agree that this was unclear and its replacement in the revised paper  (Lemma~1) has an entirely new proof.


\newpage
\subsection*{Review 2}

\textbf{Comment 2.1:}
\comment{One has a feeling (based only on visualization of very low dimensional instances) that the authors probably have a correct algorithm for their computations, but the presentation of the proof is not good. The parts of Definition 2 related to the ``orientation cone'' are wrong, and it appears that the proof of Theorem 1 is wrong, too. The proof on page 8 is hard to follow. The style of rambling notes would be better replaced by lemmas, with clearer statements of what is to be proved.}

\v
\textbf{Response 2.1:}
Definition 3 (called Definition 2 in the previous version of the paper) has been corrected in the revised paper, and the ``orientation cone'' that was previously used has been replaced with a reference to the more standard ``normal cone''. Lemma 1 (previously called Theorem 1) has been rewritten and its proof has been corrected. The proof that the reviewer refers to on page 8, which has been replaced with the proof of Theorem 4 in the revised paper, has also been completely rewritten. To simplify and clarify the arguments we have introduced two intermediate results, Lemma 2 and Lemma 3. 

\textbf{Comment 2.2:}
\comment{
Page 1, line 46: Delete the ``a'' from for a its.\\
Page 2, line 32: Delete the extra ``the'' in the the.\\
Page 2, line 37: only on $\Omega$.\\
Page 2, paragraph from line 32 to line 43: Some sentences here link unrelated definitions by commas when it would be better to separate them into smaller sentences.}

\v
\textbf{Response 2.2:}
Done

\v
\textbf{Comment 2.3:}
\comment{
Page 3, line 8: ``\dots given by the graph of a continuous function.'' What function is this?}

\v
\textbf{Response 2.3:}
We agree that this statement was unclear and we have removed it from the revised paper.

\v
\textbf{Comment 2.4:}
\comment{
Page 3, line 17: then there exists}

\v
\textbf{Response 2.4:}
Corrected.

\v
\textbf{Comment 2.5:}
\comment{
Page 3, line 47: If the containment goes the other way, this is called a convex set-valued function. Wasowicz, in arxiv.org/pdf/1611.02584.pdf, calls set-valued functions with the containment as in this Definition concave.
}

\v
\textbf{Response 2.5:}
We thank the reviewer for pointing out this reference, which is included in the revised paper.

\v
\textbf{Comment 2.6:}
\comment{
Page 4, line 22: The orientation cone defined here is not a cone if $z$ is a vertex of $\mathcal W(p)$.
}

\v
\textbf{Response 2.6:}
The orientation cone in the previous version of the paper has been replaced with a reference to the normal cone in Definition 3.

\v
\textbf{Comment 2.7:}
\comment{
Page 4, line 28: Why should it contain “all linear combinations thereof?”
}

\v
\textbf{Response 2.7:}
This statement has been removed from Remark~1, which now refers to the normal cone.

\v
\textbf{Comment 2.8:}
\comment{
Page 4, line 34: I do not believe that Theorem 1 is true with the given definition of $\mathcal N$. If $\mathcal N$ were the normal cone of $\mathcal W(p)$ at $z$, then it might be true.\\
Page 5, line 1: Why is it necessary to find a vector in the intersection of these two ``cones?'' Parametric convexity is what was to be proved, and that has been done.\\
Page 5, line 13: Why does it follow that $\mathcal W$ cannot be parametrically convex? If you let $\mathcal W(y) = [0,1] \times [0,1]$ for all $y \in [0,1]$, $z_1 = (1,1/2)$ and $z_2 = (1/2,1)$, then $\mathcal N_{\mathcal W} (0,z_1) \cap \mathcal N_{\mathcal W}(1,z_2) \neq \emptyset$ but $(1/2)(0,z_1) + (1/2)(1, z_2) \in int(\mathscr G(\mathcal W))$ and $\mathcal W$ is parametrically convex.}

\v
\textbf{Response 2.8:}
The reviewer is correct that the previous version of the paper had an error in Theorem~1. This has been replaced with Lemma 1 in the revised paper, which is stated in terms of normal cones. The proof of Lemma 1 is has been completely reconstructed, and it does not suffer from the problems that the reviewer mentions here.

\v
\textbf{Comment 2.9:}
\comment{
Page 5, line 42: It would be useful to give a definition of the usual Pontryagin difference, in order to contrast it with the parametric one.
}

\v
\textbf{Response 2.9:}
This is now done in the section on notation  at the end of the Introduction 


\v
\textbf{Comment 2.10:}
\comment{
 Page 6, line 32: ``this is always possible if $S' = S \ominus \mathcal W(S)$ has a non- empty interior.'' This is a condition for constructing $S$, but it involves $S$ itself. Why can’t $\{z_1\} \oplus \mathcal W(z_1)$ be contained in the interior of $\{z_2\} \oplus \mathcal W(z_2)$, for example?
}

\v
\textbf{Response 2.10:}
This part of the proof shows that parametric convexity is necessary for $ S \ominus \mathcal W(S)$ to be convex for all convex sets $S$ (this is emphasised more clearly in the revised paper), therefore we only need to be able to construct one convex $S$ for which $ S \ominus \mathcal W(S)$ is non-convex and we are free to choose $z_1$ and $z_2$.

\v
\textbf{Comment 2.11:}
\comment{
Page 7, line 11: It seems like the boundary would still be locally affine if $B$ contained one nonzero row, or if $B$ were a rank one matrix.
}

\v
\textbf{Response 2.11:}
We agree and have corrected this statement (Theorem 3) in the revised paper.

\v
\textbf{Comment 2.12:}
\comment{Page 7, line 11: Why are the roles of $b$ and $c$ interchanged between Corollary 2 and Corollary 3?}

\v
\textbf{Response 2.12:}
This has been changed in the revised paper so that parameters $b$ and $c$ have the same roles throughout Sections 4 and 5.

\v
\textbf{Comment 2.13:}
\comment{
Page 7, line 21: It would be helpful to get a definition of a generic piecewise affine polytopic set-valued map. Does it only mean that each $\mathcal W(p)$ is simple? The statement in the introduction is vague.}

\v
\textbf{Response 2.13:}
This is now defined (Definition 1) in Section 2

\v
\textbf{Comment 2.14:}
\comment{
Page 7, line 30: What is the index set for $k$?
}

\v
\textbf{Response 2.14:}
The index set for $k$ is now stated in the revised paper.

\v
\textbf{Comment 2.15:}
\comment{
Page 7, line 45: $\mathscr G(\mathcal W)$ consists of pairs $(p,z)$.}


\v
\textbf{Response 2.15:}
This has been corrected.

\v
\textbf{Comment 2.16:}
\comment{Page 8, line 18: What does it mean for two vertices to merge?}

\v
\textbf{Response 2.16:}
This terminology was rather imprecise and the revised does not refer to vertices merging, but refers to changes in the number of vertices of $\mathcal W(p)$ instead.

\v
\textbf{Comment 2.17:}
\comment{Page 8, line 20: Should it be for $p$ such that $fp = g$?}

\v
\textbf{Response 2.17:}
The revised paper provides a new proof of Theorem 4, which no longer uses this argument.


\v
\textbf{Comment 2.18:}
\comment{Page 9, line 7: W(p) is a set-valued map. W(p1) and W(p2) are sets.}

\v
\textbf{Response 2.18:}
This has been corrected in the revised paper.

\v
\textbf{Comment 2.19:}
\comment{Page 11, line 8: What does admissible mean? Does $w^\ast$ have to be in $\mathcal W(p)$? Later, “feasible” is used.}

\v
\textbf{Response 2.19:}
References to admissible/feasible $w$ have been replaced with $w\in\mathcal W(p)$ in the revised paper.

\v
\textbf{Comment 2.20:}
\comment{Page 11, line 19: It would help to see a definition of the term shape.}

\v
\textbf{Response 2.20:}
The reference to ``the shape of the solution over $p$'' has been replaced with ``the dependence of the solution on $p$''.

\v
\textbf{Comment 2.21:}
\comment{Page 11, line 31: an element-wise\\
Page 13, line 11: we require a\\
Page 17, line 27: The same authors have a 2007 paper in Journal of Optimization Theory and Applications with almost the same name. Presumably it is an improved and more accessible version.}

\v
\textbf{Response 2.20:}
Done.

\newpage

\subsection*{Review 3 (included in the Editor's email)}

\textbf{Comment:}
\comment{
An informal quick comment by another expert may be of interest,
and I include it in case it might be helpful:

"I went quickly through the paper. Certainly, it contains interesting results, but the
way it is written hides the most interesting result in my opinion. The most interesting
result for me is at the last sentence of Section 3, "Pontriyagin difference \dots\ is itself
polyhedral." This statement gives a strong incentive to find an algorithm. Without this,
one cannot hope to compute the difference. Why don't the authors write this at the
beginning and motives the notion such as continuity and parametric convexity of
set-valued maps as they are necessary conditions for the polyhedrality of the difference?
Anyway, the paper is pretty good, assuming the proofs are correct."
}

\v
\textbf{Response:}
We thank the reviewer for this positive assessment. We feel that the property that the parametric Pontryagin difference between a convex polyhedral set and a piecewise affine polyhedral set-valued map is polyhedral is not so surprising, given that the usual Pontryagin difference between two polyhedral sets is polyhedral. We are also unclear about what the reviewer means by an incentive to find an algorithm [for computing the parametric Pontryagin difference], since the paper does discuss the computational tools needed to determine the parametric Pontryagin difference in Section 5. In fact two computational methods are proposed, based respectively on vertex enumeration and projection.


\end{document}  